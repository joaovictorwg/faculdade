\documentclass[12pt]{exam}
\usepackage[utf8]{inputenc}
%\usepackage[portuguese,brazil]{babel}
%\usepackage[margin=.75in]{geometry}
\usepackage{amsmath,amssymb}
\usepackage{multicol}
\usepackage{multirow}
\usepackage{float}
\usepackage{enumerate}
\usepackage{array}
\usepackage{graphicx}% delete the demo option in your actual code
\usepackage{xcolor}
\usepackage{tikz}

% \usepackage[usenames,dvipsnames]{pstricks}
\usepackage{epsfig}
\usepackage{pst-grad} % For gradients
\usepackage{pst-plot} % For axes
\usepackage[space]{grffile} % For spaces in paths
\usepackage{etoolbox} % For spaces in paths
% \makeatletter % For spaces in paths


\pointpoints{ponto}{pontos}
\hpword{Pontos:}
\vpword{Pontos}
\htword{Total}
\vtword{Total:}
\vsword{Resultado}


\SolutionEmphasis{\color{red}}
\CorrectChoiceEmphasis{\bf \color{red}}
\renewcommand{\solutiontitle}{\noindent\textbf{Solução:}\enspace}


\begin{document}


\printanswers

\begin{center}
\textbf{{\Huge Atividades}}
\end{center}

\begin{questions}

\section*{Simplex}

\question Seja o Problema de Programação Linear a seguir:
	
	\begin{table}[H]
		\centering
			\begin{tabular}{c c c r c r c c }
			($PPL$) :&  Maximizar & $x_1$ & + & $x_2$   &  & &	\\
			& sujeito a: & $x_1$ & + & $4x_2$  & $\leq $ &	 4 \\
			 & & $3x_1$ & $+$ & $x_2$ &  = &	 1 \\
			 & & $x_1$ & , & $x_2$ &  $\geq$ &	 0
			\end{tabular}
	\end{table}	
	Resolva o PPL pelo método do simplex.
	
	\begin{solution}

	\begin{equation*}	
			\begin{array}{c r r c r c c c }
			(PPL) :& \text{Maximizar} & x_1  & + &  x_2    &  & &	\\
			           & \text{sujeito a:}  & x_1 & + & 4x_2  &  \leq  &	 4 \\
			           &                              & 3x_1 & + & x_2 &  = &	 1 \\
			           &                              & \multicolumn{6}{c}{x_1,~x_2 \geq 0}
			\end{array}	
	\end{equation*}
Adicionando variável de folga:
	\begin{equation*}	
			\begin{array}{c r r c r crc c c }
			(PPL) :& \text{Maximizar} & x_1   & + &  x_2   & + & 0x_3 &  & &	\\
			           & \text{sujeito a:}  & x_1   & + & 4x_2  & + &  x_3 &  =  &	 4 \\
			           &                              & 3x_1 & + & x_2    &     &        &  =  &	 1 \\
			           &                              & \multicolumn{6}{c}{x_1,~x_2,~x_3 \geq 0}
			\end{array}	
	\end{equation*}
	
Resolvendo o PPL:
	$$A = (a_1~a_2~a_3) =\begin{pmatrix}
	1 & 4 & 1  \\ 
	3 & 1 & 0  \\ 
	\end{pmatrix},~
	b = \begin{pmatrix}
	4 \\ 
	1
	\end{pmatrix},~c = (1~1~0).$$
	$$I_B = \{1,~3\},~~I_N = \{2\},$$
	$$B(1) = 1,~ B(2) = 3$$
	$$B = (a_1~a_3) =\begin{pmatrix}
	1 & 1 \\
	3 & 0
	\end{pmatrix},~\text{logo } B^{-1} = \begin{pmatrix}
	0 & \frac{1}{3} \\
	1 & -\frac{1}{3}
	\end{pmatrix},~c_B = (1~0),$$
	$$~u = c_BB^{-1} = (1~0)\begin{pmatrix}
	0 & \frac{1}{3} \\
	1 & -\frac{1}{3}
	\end{pmatrix} = (0~\frac{1}{3}),$$
	
	$$ \overline{x}_B =B^{-1}b = \begin{pmatrix}
	\overline{x}_1 \\ 
	\overline{x}_3 
	\end{pmatrix} = \begin{pmatrix}
	0 & \frac{1}{3} \\
	1 & -\frac{1}{3}
	\end{pmatrix} \begin{pmatrix}
	4 \\ 
	1 \\ 
	\end{pmatrix} = \begin{pmatrix}
	\frac{1}{3} \\ 
	\frac{11}{3} 
	\end{pmatrix}$$
	$$\overline{z} = c_BB^{-1}b = ub = (0~\frac{1}{3}) \begin{pmatrix}
	4 \\ 
	1 
	\end{pmatrix} = \frac{1}{3}$$

	$$z_2 = ua_2 = (0~\frac{1}{3})\begin{pmatrix}
	4 \\ 
	1 
	\end{pmatrix} = \frac{1}{3}~\quad \Rightarrow \quad z_2 - c_2 = \frac{1}{3} - 1 = -\frac{2}{3} \leq 0$$


	$$y_2 = B^{-1}a_2 = \begin{pmatrix}
	0 & \frac{1}{3} \\
	1 & -\frac{1}{3}
	\end{pmatrix} \begin{pmatrix}
	4 \\ 
	1
	\end{pmatrix}=  \begin{pmatrix}
	\frac{1}{3} \\ 
	\frac{11}{3} 
	\end{pmatrix} $$

Tomemos, $x_2$ para ter seu valor aumentado, isto é, faremos a coluna $a_2$ entrar na nova base. Como $L_1 = \{1, 2\}$, pois $y_{12} = \frac{1}{3}$, $y_{22} = \frac{11}{3}$, passaremos a calcular $\alpha_2$:
$$\alpha_2 = \min\left\lbrace\frac{\overline{x}_{B(1)}}{y_{12}},\frac{\overline{x}_{B(2)}}{y_{22}}\right\rbrace  = \min\left\lbrace\frac{\frac{1}{3}}{\frac{1}{3}}, \frac{\frac{11}{3}}{\frac{11}{3}} \right\rbrace = \frac{1}{3} = \frac{\overline{x}_{B(2)}}{y_{22}},$$
logo $a_{B(2)}$ deixará a base, sendo substituída pela coluna $a_2$.

2ª Solução básica:
	$$I_B = \{1,~2,\},~~I_N = \{3\},$$
	$$B(1) = 1,~ B(2) = 2$$
	$$B = (a_1~a_2) =\begin{pmatrix}
	1 & 4 \\
	3 & 1
	\end{pmatrix},~\text{logo } B^{-1} = \begin{pmatrix}
	-\frac{1}{11} & \frac{4}{11} \\
	\frac{3}{11} & -\frac{1}{11}
	\end{pmatrix},~c_B = (1~0),$$
	$$~u = c_BB^{-1} = (1~1)\begin{pmatrix}
	-\frac{1}{11} & \frac{4}{11} \\
	\frac{3}{11} & -\frac{1}{11}	
	\end{pmatrix} = (\frac{2}{11}~\frac{3}{11}),$$
	
	$$ \overline{x}_B =B^{-1}b = \begin{pmatrix}
	\overline{x}_1 \\ 
	\overline{x}_2 
	\end{pmatrix} = \begin{pmatrix}
	-\frac{1}{11} & \frac{4}{11} \\
	\frac{3}{11} & -\frac{1}{11}
	\end{pmatrix} \begin{pmatrix}
	4 \\ 
	1 \\ 
	\end{pmatrix} = \begin{pmatrix}
	0 \\ 
	1
	\end{pmatrix}$$

	$$\overline{z} = c_BB^{-1}b = ub = (\frac{2}{11}~\frac{3}{11}) \begin{pmatrix}
	4 \\ 
	1 
	\end{pmatrix} = 1$$
	
	$$z_3 = ua_3 = (\frac{2}{11}~\frac{3}{11})\begin{pmatrix}
	1 \\ 
	0 
	\end{pmatrix} = \frac{2}{11}~\quad \Rightarrow \quad z_3 - c_3 = \frac{2}{11} - 0 = \frac{2}{11} \geq 0$$
	
	Como $z_j - c_j \geq 0$, $\forall j \in I_N$ , esta solução básica (2ª solução) é ótima para o PLL. \\ 
Então $x_1 = 0 ,~x_2 = 1,~x_3 = 0$ é uma solução ótima, fornecendo $z^{*} = 1$. 	\end{solution}	

\end{questions}
	

\end{document}
